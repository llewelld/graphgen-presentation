% This is LLNCS.DEM the demonstration file of
% the LaTeX macro package from Springer-Verlag
% for Lecture Notes in Computer Science,
% version 2.4 for LaTeX2e as of 16. April 2010
%
\documentclass[envcountsame]{llncs}
%
%\usepackage{makeidx}  % allows for indexgeneration


\usepackage{hyperref}
\usepackage{amsmath}
\usepackage[pdftex]{graphicx}
\usepackage{tabularx}
\graphicspath{{./figures/}}
\DeclareGraphicsExtensions{.pdf,.jpeg,.png,.jpg}
\usepackage{enumitem}
\usepackage{subcaption}
\usepackage{multirow}
\usepackage{marginnote}
\usepackage{yhmath}
\usepackage{eqnarray}

\def\etal{{\it et al.}}
\def\etc{{\it etc.}}
\def\eg{{\it e.g.}}
\def\ie{{\it i.e.}}
\def\cf{{\it cf.}}
\def\qv{{\it q.v.}}
\def\qqv{{\it qq.v.}}
\def\st{s.t.\ }
\def\concat{\mathbin{|}}

\newtheorem{notation}{Notation}[section]
\def\cc{\tt}

\newcolumntype{Y}{>{\centering\arraybackslash}X}
\newcommand{\comment}[1]{\marginnote{\textit{\textbf{\scriptsize{#1}}}}}
\newcolumntype{e}{@{\qquad}}
\newcommand{\rotr}[1]{\multicolumn{1}{c|}{\rotatebox[origin=l]{90}{#1}}}
\newcolumntype{C}[1]{>{\centering\let\newline\\\arraybackslash\hspace{0pt}}m{#1}}

%\makeindex

%
\begin{document}

\spnewtheorem{claimnum}{Claim}{\itshape}{\rmfamily}
%
\frontmatter          % for the preliminaries
%
\pagestyle{headings}  % switches on printing of running heads
\addtocmark{Example paper} % additional mark in the TOC
%
\mainmatter              % start of the contributions
%
\title{Automatically Generating Visualisations}
%
\titlerunning{Automatically Generating Visualisations}  % abbreviated title (for running head)
%                                     also used for the TOC unless
%                                     \toctitle is used
%
\author{David Llewellyn-Jones}
%
\authorrunning{David Llewellyn-Jones} % abbreviated author list (for running head)
%
%%%% list of authors for the TOC (use if author list has to be modified)
\tocauthor{David Llewellyn-Jones}
%
\institute{Computer Laboratory, University of Cambridge, Cambridge, UK,\\
\email{David.Llewellyn-Jones@cl.cam.ac.uk},\\
\texttt{http://www.flypig.co.uk}}

\maketitle              % typeset the title of the contribution

\begin{abstract}
Automatically generating graphs and visualisations of experimental data can be a powerful way to encourage experimentation. It also ensures graphs are up-to-date and consistent across all documents containing them.
\end{abstract}


\section{Introduction}

There are a myriad great tools for generating graphs, images and figures for use in scientific publications. Just a few examples include R \cite{introtor}, Matlab/Octave \cite{octave} and Matplotlib \cite{matplotlib}. The graphs shown in figure \ref{fig:autogen} were generated using Matplotlib.pyplot, which provides an interface to Matplotlib that Matlab/Octave users will find familiar. In the event the data changes, the graphs can be automatically regenerated and are inserted directly into this document without the need for manual intervention.

\begin{figure}[h!]
\centering
\begin{subfigure}[t]{0.49\textwidth}
\centering
\includegraphics[width=1.0\textwidth]{strange03}%
\caption{Strange attractor.}
\label{fig:strange}
\end{subfigure}
\begin{subfigure}[t]{0.49\textwidth}
\centering
\includegraphics[width=1.0\textwidth]{automark01}%
\caption{Automark optimisationo.}
\label{fig:automar}
\end{subfigure}
\caption{Graphs generated and inserted automatically.}
\label{fig:autogen}
\end{figure}


%
% ---- Bibliography ----
%

\bibliographystyle{splncs03}
\bibliography{main}



%\clearpage
%\addtocmark[2]{Subject Index} % additional numbered TOC entry
%\markboth{Subject Index}{Subject Index}
%\renewcommand{\indexname}{Subject Index}
%\printindex
\end{document}
